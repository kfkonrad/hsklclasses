\addsec{Kurzüberblick}

Um eine Studienarbeit nach den Vorgaben von Prof.~Adrian~Müller zu schreiben,
kann man folgende Schritte befolgen:

\begin{enumerate}
	\item Die Dokumentenklasse laden:\\
		\verb!\listfiles\documentclass[shortprelims,oneside,openany,!\\
		\verb!index=totoc,listof=totoc,bibliography=totoc]{hsklbook}!
	\item Die Anpassungen für das Literaturverzeichnis laden:\\
		\verb!\usepackage{bonmuellerstyle}!
	\item Die Daten für das Titelblatt mit den folgenden Befehlen angeben:\\
		\begin{minipage}{\textwidth}
			\begin{multicols}{2}
				\verb!\hslogofile{path/to/file}!\\
				\verb!\hstitle{Mein toller Titel}!\\
				\verb!\hssubtitle{Ein cooler Untertitel}!\\
				\verb!\hsdate{10. Januar 2038}!\\
				\verb!\hsyear{2038}!\\
				\verb!\hsmajor{Medieninformatik}!\\
				\verb!\hsmatriculation{876543}!\\
				\verb!\hsauthor{Kevin Konrad}!
			\end{multicols}
		\end{minipage}

		\verb!\hssupervisorname{Prof.~Adrian~Müller, PMP, PSM-I, CSM}!
	\item Das Babel-Paket für deutsche Sprachanpassungen und Biblatex mit
		Einstellungen nach A.~Müllers Vorgaben werden automatisch eingebunden. Nur
		den Namen der \verb!.bib!-Datei muss man noch angeben:\\
		\verb!\bibliography{literatur}!\\
		Die
		Pakete für das Handling von Schrift- \& Eingabekodierung muss man selbst
		einbinden:\\
		Windows: \verb!\usepackage[T1]{fontenc} \usepackage[ansinew]{inputenc}!\\
		Linux: \verb!\usepackage[T1]{fontenc} \usepackage[latin1]{inputenc}!\\
		macOS: \verb!\usepackage[T1]{fontenc} \usepackage[applemac]{inputenc}!\\
		utf8: \verb!\usepackage[T1]{fontenc} \usepackage[utf8]{inputenc}!\\
		mit \hologo{XeLaTeX} oder \hologo{LuaLaTeX}: \verb!\usepackage{fontspec}!
	\item \verb!\begin{document}!\\
		\verb!\hsfrontmatter\hsmainmatter!\\
		Damit ist jetzt der Titel erzeugt und es geht an die eigentliche Arbeit.\\
		Hier setzt man dann die Kapitel ein, etwa mit \verb!\input{kapitel1}!.\\
		Zum Schluss wird nur noch mit\\
		\verb!\hsbackmatter!\\
		\verb!\end{document}!\\
		das Literaturverzeichnis erzeugt. Bei Bedarf können hinter
		\verb!\hsbackmatter! mit\\\verb!\addchap{Anhang}\addsec{Beispielanhang 1}!
		Anhänge angelegt werden.
\end{enumerate}

Diese Befehle gibt es auch schon fertig verpackt in einem Minimalbeispiel. Mehr
Informationen über die sonstigen Fähigkeiten von \HSKLbook
stehen im weiteren Verlauf der Dokumentation, mit allen Features ist
\HSKLbook auch für eine Bachelorthesis geeignet.

Für Windows-Anwender empfehle ich die IDE \TeX{}nikCenter\footnote{Download
unter \url{http://www.texniccenter.org/download/}}. Sie ist einfach zu
bedienen und bietet Bugtracking bis zur Ursprungszeile im Code. Um damit die
STUA zu kompilieren, muss man das mitgelieferte \verb!Ausgabeprofil_STUA.tco!
importieren. Dazu bei geöffnetem \TeX{}nikCenter \verb!Alt + F7! drücken. Auf
\verb!Importieren! klicken, erneut auf \verb!Importieren! klicken, das neue
Profil \verb!Latex => PDF - STUA! auswählen und mit \verb!OK! bestätigen.

Für Mac- \& Linuxnutzer gibt es ein Makefile. Der Name
der Haupt-\TeX-Datei muss dafür \verb!stua.tex! und der Name der
\verb!.bib!-Datei \verb!literatur.bib! lauten. Die Namen sind alternativ im
Makefile änderbar.

Was unterscheidet diese Vorlage von den Standardklassen oder den reinen
\hologo{KOMAScript}-Klassen? Mit \HSKLbook ist die Formatierung der
Arbeit schon von Beginn an abgeschlossen. Das Design ist abgesehen von der
Schriftart (hier wird Computer Modern, die \LaTeX-Standardschriftart, benutzt)
an das Corporate Design der Hochschule Kaiserslautern angelehnt. Wer alle
Befehle des obigen Beispiels ausfällt, braucht sich auch um die Formatierung der
Titelseite keine Gedanken mehr zu machen.

Für eine Bachelorarbeit braucht man die obige Klassenoption \verb!oneside!
(1. Listenpunkt, 2. Zeile) nur durch \verb!twoside=semi! zu ersetzen und die
Option \verb!shortprelims! zu entfernen. Dadurch wird auch die
Copyrightanmerkung erzeugt.

Die Möglichkeit je ein Glossar, Symbol- \&
Abkürzungsverzeichnis zu erzeugen bietet \HSKLbook ebenfalls.\footnote{Für
exemplarische Einträge das Paket \texttt{AbkSymGlo.sty} einbinden und mit den
darin vermerkten Befehlen im Fließtext einsetzen. Gerne auch eigene Einträge
schreiben :)} Außerdem kann ein Sperrvermerk generiert werden und sogar eine
Trennung von Online- \& Offlinequellen ist möglich.\footnote{Hierzu die
Klassenoption \texttt{splitbib} angeben und jeden online-Eintrag mit dem Keyword
\enquote{Internet} versehen.}
