\section{Allgemeiner Aufbau der Dokumentenklassen}

Diese Dokumentation behandelt die Dokumentenklassen \HSKLarticle,
\HSKLreport und \HSKLbook. Diese basieren auf den korrespondieren
KOMAScript-Klassen \verb!scrartcl!, \verb!scrreprt! sowie \verb!scrbook!, wobei
auf Kurznammen wie \verb!XYreprt! zugunsten der Lesbarkeit und Merkbarkeit
verzichtet wurde.

Der Großteil der Ergänzungen zu den KOMAScript-Klassen dient der Anpassung des
Dokumentenlayouts. Jeder Befehl aus den KOMAScript-Klassen ist auch in den
jeweiligen \HSKL-Klassen verfügbar, die umfangreiche Dokumentation von KOMAScript
gibt Aufschluss über deren gesamte Funktionalität. Die jeweils aktuelle Fassung
findet sich unter \url{http://mirrors.ctan.org/macros/latex/contrib/koma-script/doc/scrguide.pdf}.

\subsection{Wichtige Funktionen der \HSKL-Klassen}
Die deutsche Sprachanbindung durch \verb!babel! ist automatisiert und
um einige fehlende Übersetzungen ergänzt worden. Bei Bedarf kann \verb!babel!
wie gewohnt mit benutzerdefinierten Sprachlisten wie etwa \verb![ngerman,english]! eingebunden werden.

Der Mathematiksatz ist mithilfe von \AmS\hspace{1pt}math optimiert. Zum Thema
zeitgemäßer Mathematiksatz empfiehlt der Autor das \LaTeX\ Math Cheat Sheet,
das unter \url{https://github.com/kfkonrad/mathcheat/blob/master/mathcheat.pdf}
heruntergeladen werden kann. In dem Cheat~Sheet stehen auf einer Seite alle
wichtigen, modernen Befehle, die man für den Mathematiksatz in \LaTeX\ benötigt.

Um einen höheren vertikalen Abstand zwischen zwei Elementen zu erzeugen gibt es
den Befehl \verb!\extravspace!. Anders als bei Befehlen wie \verb!\vspace{1cm}!
wird mit \verb!\extravspace! kein fester Abstand erzeugt, sondern einer, der vom
aktuell eingestellten Zeilenabstand abhängig ist. Dadurch ist auch nach Änderung
des Zeilenabstandes, der Schriftart oder Schriftgröße eine einheitliche Ausgabe
gesichert.
Der Zeilenabstand kann mit dem Befehl \verb!\linespread{1.5}! angepasst werden,
Standardwert ist 1.2.

Für Tabellen gibt es eine angepasste Umgebung namens \verb!hstable!. Sie ist
wie \verb!tabular! zu benutzen und kann mehrere Seiten überspannen. Für
\verb!hstable! steht zudem die Funktionalität der Pakete
\verb!longtable!\footnote{Dokumentation unter
\url{http://mirrors.ctan.org/macros/latex/required/tools/longtable.pdf}}
und \verb!booktabs!\footnote{Deutschsprachige Dokumentation unter
\url{http://mirrors.ctan.org/macros/latex/contrib/booktabs-de/booktabs-de.pdf}}
zur Verfügung. Um optisch anspruchsvolle und typographisch saubere Tabellen zu
erzeugen wird \emph{dringend} empfohlen, wenigstens die Mittel des
\verb!booktabs! Paketes zu verwenden. Die Dokumentation erklärt die Anwendung
und die Gründe für die zugehörigen Designentscheidungen ausführlich.
