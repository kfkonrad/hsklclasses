\subsection{Klassenoptionen}
Zusätzlich zu sämtlichen Klassenoptionen (wie etwa \verb!paper=a4!) aus
\verb!scrbook! unterstützt \verb!hsklbook! weitere Optionen, die alle als
Schalter mit wahr/falsch-Werten (Englisch: \verb!true! bzw. \verb!false!)
realisiert sind. Nachfolgend wird der Zweck jeder Option in alphabetischer
Reihenfolge erläutert, der Standardwert steht \emph{kursiv} am Seitenrand.

\iffalse%Hier stehen die Optionen in natürlich gewachsender Abfolge
\parasection{dedication} Die Ausgabe der Widmung nach dem Titel de-/aktivieren.\dtrue
\parasection{restricitonnote} Einen Sperrvermerk generieren. \dfalse
\parasection{shortprelims} Einen verkürzten Titel ohne Copyright-Seite erzeugen. \dfalse
\parasection{titlepage} Die Titelei erzeugen. \dtrue
\parasection{internetbibliography} Die \dfalse Literatur nach Internet- \& sontigen Quellen trennen. Hierzu muss das Keyword \verb!Internet! in jedem \verb!.bib!-Eintrag stehen, der als Internetliteratur gelistet werden soll.
\parasection{listoftables} Ein Tabellenverzeichnis erzeugen. \dtrue
\parasection{listoffigures} Ein Abbildungsverzeichnis erzeugen. \dtrue
\parasection{titlepage-showsupervisor} Die Betreuung auf der Titelseite anzeigen. \dtrue
\parasection{titlepage-showmatriculation} Die Matrikelnummer auf der Titelseite anzeigen. \dtrue
\parasection{copyright-showname} Den Namen auf der Copyright-Seite anzeigen. \dtrue
\parasection{copyright-showaddress} Die Adresse auf der Copyright-Seite anzeigen. \dtrue
\parasection{copyright-showpermanentcontact} Den permanenten Kontakt (E-Mail) auf der Copy\-right-Seite anzeigen. \dtrue
\parasection{copyright-multiauthor} Eine Tabelle \dfalse für mehrere Autoren auf der Copyright-Seite anzeigen.
\parasection{multiauthor} Komfortschalter, \dfalse der die notwendigen Einstellungen für Dokumente mit mehreren Autoren in einer Anweisung setzt:\\\verb!titlepage-showmatriculation=false!\\\verb!copyright-showname=false!\\\verb!copyright-showaddress=false!\\\verb!copyright-showpermanentcontact=false!\\\verb!copyright-multiauthor=true!
\parasection{custombiblatex} Wenn nicht \dfalse die Standardeinstellungen der \HSKLbook-Klasse zu biblatex gewünscht sind, kann mit diesem Schalter auch die Einbindung unterdrückt werden. Dann ist es jedoch \emph{unbedingt} notwendig, biblatex selbst mit den gewünschten (abweichenden) Optionen zu laden!! Der Standardaufruf zu biblatex lautet:\\\verb!\usepackage[style=authoryear-icomp,!\\\verb!isbn=true,!\\\verb!pagetracker=true,!\\\verb!maxbibnames=50,!\\\verb!maxcitenames=2,!\\\verb!autocite=inline,!\\\verb!block=space,!\\\verb!backref=true,!\\\verb!backrefstyle=three+,!\\\verb!date=short,!\\\verb!url=true,!\\\verb!backend=biber]{biblatex}!
\parasection{smallereditionnotes} Bewirkt eine \dfalse Verkleinerung der Schriftgröße der Publishing-Angaben unten auf der Titelrückseite.
\fi%Und hier alphabetisch

\parasection{copyright-multiauthor} Eine Tabelle \dfalse für mehrere Autoren auf der Copyright-Seite anzeigen.
\parasection{copyright-showaddress} Die Adresse auf der Copyright-Seite anzeigen. \dtrue
\parasection{copyright-showname} Den Namen auf der Copyright-Seite anzeigen. \dtrue
\parasection{copyright-showpermanentcontact} Den permanenten Kontakt (E-Mail) auf der Copy\-right-Seite anzeigen. \dtrue
\parasection{custombiblatex} Wenn nicht \dfalse die Standardeinstellungen der \HSKLbook-Klasse zu biblatex gewünscht sind, kann mit diesem Schalter auch die Einbindung unterdrückt werden. Dann ist es jedoch \emph{unbedingt} notwendig, biblatex selbst mit den gewünschten (abweichenden) Optionen zu laden!! Der Standardaufruf zu biblatex lautet:\\\verb!\usepackage[style=authoryear-icomp,!\\\verb!isbn=true,!\\\verb!pagetracker=true,!\\\verb!maxbibnames=50,!\\\verb!maxcitenames=2,!\\\verb!autocite=inline,!\\\verb!block=space,!\\\verb!backref=true,!\\\verb!backrefstyle=three+,!\\\verb!date=short,!\\\verb!url=true,!\\\verb!backend=biber]{biblatex}!
\parasection{dedication} Die Ausgabe der Widmung nach dem Titel de-/aktivieren.\dtrue
\parasection{internetbibliography} Die \dfalse Literatur nach Internet- \& sontigen Quellen trennen. Hierzu muss das Keyword \verb!Internet! in jedem \verb!.bib!-Eintrag stehen, der als Internetliteratur gelistet werden soll.
\parasection{listoffigures} Ein Abbildungsverzeichnis erzeugen. \dtrue
\parasection{listoftables} Ein Tabellenverzeichnis erzeugen. \dtrue
\parasection{multiauthor} Komfortschalter, \dfalse der die notwendigen Einstellungen für Dokumente mit mehreren Autoren in einer Anweisung setzt:\\\verb!titlepage-showmatriculation=false!\\\verb!copyright-showname=false!\\\verb!copyright-showaddress=false!\\\verb!copyright-showpermanentcontact=false!\\\verb!copyright-multiauthor=true!
\parasection{restricitonnote} Einen Sperrvermerk generieren. \dfalse
\parasection{shortprelims} Einen verkürzten Titel ohne Copyright-Seite erzeugen. \dfalse
\parasection{smallereditionnotes} Bewirkt eine \dfalse Verkleinerung der Schriftgröße der Publishing-Angaben unten auf der Titelrückseite.
\parasection{titlepage-showmatriculation} Die Matrikelnummer auf der Titelseite anzeigen. \dtrue
\parasection{titlepage-showsupervisor} Die Betreuung auf der Titelseite anzeigen. \dtrue
\parasection{titlepage} Die Titelei erzeugen. \dtrue
