\subsection{Neue Befehle}
Die gesamte übliche Titelei wird automatisiert erstellt. Die nachfolgenden
Befehle dienen zur Anpassung von Namen, Thema und Ähnlichem in der Titelei und
dem Sperrvermerk. Die eingetragenen Werte sind exemplarisch.

Wenn durch einzelne Optionen bestimme Ausgaben unterdrückt werden, so können die
Daten dazu (etwa copyright-showpermanentcontact und \verb!\hspermanentcontact!)
dennoch angegeben werden. So lässt sich bei Bedarf eine Option de-/aktivieren,
ohne dass weitere Änderungen am Dokument nötig sind.

\iffalse%Hier stehen die Befehle in natürlich gewachsender Abfolge
\neubefehl{hslogofile}{Pfad zum Hochschullogo festlegen.}{pfad/zum/logo}
\neubefehl{hstitle}{Titel der Arbeit.}{Titelchen}
\neubefehl{hssubtitle}{Untertitel der Arbeit.}{Mein genialer Untertitel}
\neubefehl{hsdate}{Datum der Arbeit. Falls nicht angegeben wird das aktuelle Datum verwendet.}{10.~Januar~2038}
\neubefehl{hsyear}{Jahr in dem die Arbeit geschrieben wurde. Falls nicht angegeben, wird das aktuelle Jahr verwendet.}{2038}
\neubefehl{hsmajor}{Studiengang.}{Studiengängchen}
\neubefehl{hsmatriculation}{Matrikelnummer.}{0815123}
\neubefehl{hsauthor}{Voller Name des Autors.}{Hugo Nämchen}
\neubefehl{hsaddress}{einzeilige Adresse.}{Beispielstraße 1, 12345 Teststadt}
\neubefehl{hspermanentcontact}{Permanenter Kontakt wie z.B. E-Mail.}{mailchen@example.com}
\neubefehl{hsaddmultiauthor}{Neuen Autoreneintrag für Copyrightseite einfügen. Die Reihenfolge der 4 Argumente ist \ttfamily Voller Name\normalfont, \ttfamily Matrikelnummer\normalfont, \ttfamily einzeilige Adresse\normalfont, \ttfamily{Per\-ma\-nent\-kon\-takt (Email)}\normalfont.}{Zu M. Beispiel\}\{0815123\}\{Teststraße 42, 12345 Dummenbach\}\\\quad\{beispiel@example.com}
\neubefehl{hsplaceofstudy}{Studienort, also Zweibrücken, Pirmasens oder Kaiserslautern.}{Studienörtchen}
\neubefehl{hsplaceofstudyaddress}{Adresse des Studienorts, einzeilig.}{Beispielstraße 1, 12345 Teststädtchen}
\neubefehl{hssupervisorname}{Name des Betreuers.}{Nämelein}
\neubefehl{hssupervisorphone}{Telefonnummer des Betreuers.}{+49 012 3456 7890}
\neubefehl{hssupervisormail}{E-Mail des Betreuers.}{naemchen@example.com}
\neubefehl{hsdistribution}{Verwendete LaTeX-Distribution. Meistens ist das \texttt{\textbackslash{}Mik\{\}TeX} oder \texttt{\textbackslash{}TeX\textasciitilde{}Live}.}{\textbackslash{}TeX\{\}lein}
\neubefehl{hsdedication}{Widmung oder Zitat.}{Das Glück ist mit den Dummen.}
\neubefehl{hskindofpaper}{Art der Arbeit: Bachelorarbeit, Praxisbericht, Hausarbeit etc.}{Arbeitchen}
\neubefehl{hsworkenvironment}{Ihre Entwicklungsumgebung bzw. Ihr Editor, zB \texttt{\textbackslash{}TeX\{\}nicCenter}, \texttt{Atom} oder \texttt{\textbackslash{}TeX\{\}works}.}{Umgebüngchen}
\neubefehl{hsaddsupervisor}{Einen weiteren Betreuer mit vollem Namen, Telefonnnummer und E-Mail hinzufügen.}{Zu M. Beispiel\}\{+49 081 5123\}\{beispiel@example.com}
\neubefehl{hscompany}{Name des Unternehmens mit Rechtsform.}{Firma GerneGroß}
\neubefehl{hsrestrictedchapters}{Vom Sperrvermerk betroffene Kapitel hier benennen, auch die Anhänge.}{1, 2, 3, 5, 8, 12, Anhang B}
\neubefehl{hsfaculty}{Fachbereich.}{Fachbereichelchen}
\neubefehl{hsrestrictionduration}{Dauer der Sperre in Jahren.}{3}
\neubefehl{hsprintcountry}{Land, in dem die Arbeit/das Dokument gedruckt wurde, in Englisch anzugeben.}{Germany}
\neubefehl{hsfurthereditionnotes}{Zusätzliche editorielle Angaben zur Dokumentenversion, einer ISBN/ISSN/DOI o.\"A. falls vorhanden. Diese Angabe erscheint auf der Titelr\"uckseite unter der Zeile \enquote{Template für die Arbeit: ...}}{Dokumentenversion: 1.2}
\fi%Und hier alphabetisch
\neubefehl{hsaddmultiauthor}{Neuen Autoreneintrag für Copyrightseite einfügen. Die Reihenfolge der 4 Argumente ist \ttfamily Voller Name\normalfont, \ttfamily Matrikelnummer\normalfont, \ttfamily einzeilige Adresse\normalfont, \ttfamily{Per\-ma\-nent\-kon\-takt (Email)}\normalfont.}{Zu M. Beispiel\}\{0815123\}\{Teststraße 42, 12345 Dummenbach\}\\\quad\{beispiel@example.com}
\neubefehl{hsaddress}{einzeilige Adresse.}{Beispielstraße 1, 12345 Teststadt}
\neubefehl{hsaddsupervisor}{Einen weiteren Betreuer mit vollem Namen, Telefonnnummer und E-Mail hinzufügen.}{Zu M. Beispiel\}\{+49 081 5123\}\{beispiel@example.com}
\neubefehl{hsauthor}{Voller Name des Autors.}{Nämchen}
\neubefehl{hscompany}{Name des Unternehmens mit Rechtsform.}{Firma GerneGroß}
\neubefehl{hsdate}{Datum der Arbeit. Falls nicht angegeben wird das aktuelle Datum verwendet.}{10.~Januar~2038}
\neubefehl{hsdedication}{Widmung oder Zitat.}{Das Glück ist mit den Dummen.}
\neubefehl{hsdistribution}{Verwendete LaTeX-Distribution. Meistens ist das \texttt{\textbackslash{}Mik\{\}TeX} oder \texttt{\textbackslash{}TeX\textasciitilde{}Live}.}{\textbackslash{}TeX\{\}lein}
\neubefehl{hsfaculty}{Fachbereich.}{Fachbereichelchen}
\neubefehl{hsfurthereditionnotes}{Zusätzliche editorielle Angaben zur Dokumentenversion, einer ISBN/ISSN/DOI o.\"A. falls vorhanden. Diese Angabe erscheint auf der Titelr\"uckseite unter der Zeile \enquote{Template f\"ur die Arbeit: ...}}{Dokumentenversion: 1.2}
\neubefehl{hskindofpaper}{Art der Arbeit: Bachelorarbeit, Praxisbericht, Hausarbeit etc.}{Arbeitchen}
\neubefehl{hslogofile}{Pfad zum Hochschullogo festlegen.}{pfad/zum/logo}
\neubefehl{hsmajor}{Studiengang.}{Studiengängchen}
\neubefehl{hsmatriculation}{Matrikelnummer.}{0815123}
\neubefehl{hspermanentcontact}{Permanenter Kontakt wie z.B. E-Mail.}{mailchen@example.com}
\neubefehl{hsplaceofstudyaddress}{Adresse des Studienorts, einzeilig.}{Beispielstraße 1, 12345 Teststädtchen}
\neubefehl{hsplaceofstudy}{Studienort, also Zweibrücken, Pirmasens oder Kaiserslautern.}{Studienörtchen}
\neubefehl{hsprintcountry}{Land, in dem die Arbeit/das Dokument gedruckt wurde, in Englisch anzugeben.}{Germany}
\neubefehl{hsrestrictedchapters}{Vom Sperrvermerk betroffene Kapitel hier benennen, auch die Anhänge.}{1, 2, 3, 5, 8, 12, Anhang B}
\neubefehl{hsrestrictionduration}{Dauer der Sperre in Jahren.}{3}
\neubefehl{hssubtitle}{Untertitel der Arbeit.}{Mein genialer Untertitel}
\neubefehl{hssupervisormail}{E-Mail des Betreuers.}{naemchen@example.com}
\neubefehl{hssupervisorname}{Name des Betreuers.}{Nämelein}
\neubefehl{hssupervisorphone}{Telefonnummer des Betreuers.}{+49 012 3456 7890}
\neubefehl{hstitle}{Titel der Arbeit.}{Titelchen}
\neubefehl{hsworkenvironment}{Ihre Entwicklungsumgebung bzw. Ihr Editor, zB \texttt{\textbackslash{}TeX\{\}nicCenter}, \texttt{Atom} oder \texttt{\textbackslash{}TeX\{\}works}.}{Umgebüngchen}
\neubefehl{hsyear}{Jahr in dem die Arbeit geschrieben wurde. Falls nicht angegeben, wird das aktuelle Jahr verwendet.}{2038}
\subsection{Weiterführende Befehle der Vorlage}

Die bisherigen Befehle dienten \enquote{nur} dazu, den Lückentext der Vorlage
mit individuellen Inhalten zu füllen. Die hier eingeführten Befehle gehen
darüber hinaus, denn mit ihnen kann die Vorlage als solches manipuliert werden.
Die Anpassung der Texte auf der Titelrückseite (Copyrightseite) und die
Veränderung der Position einzelner Elemente wie Literaturverzeichnis oder
Sperrvermerk im PDF fallen aktuell hierunter.

Ziel dieser Befehle ist es, langfristig gesehen, eine universell für jede
Hochschule/Universität verwendbare Vorlage zu schaffen.

\neubefehl{hscopyrighttext}{Der Copyrighttext.}{Keine Rechte für niemanden.}

\neubefehl{hscopyrightowner}{Copyrightinhaber, sofern von der HS KL abweichend.}{Maximilian Rechtehaber}

\neubefehl{hsfurthernotices}{Weitere Angaben unterhalb des Copyrightinhabers. Hier kann z.B. stehen, wie/ob Begriffe gegendert werden, dass Zitate als soche gekennzeichnet sind etc.}{Ich zitiere niemanden, ist alles selbst ausgedacht.}

\parasectionbig{hslayout}

\texttt{\textbackslash{}hslayout} hat 2 Argumente, das erste Argument
legt fest, an welcher Stelle ein Element verwendet werden soll, während das
zweite Argument dieses Element bestimmt.

Wenn auch nur eines der verfügbaren Elemente mit dem
\texttt{\textbackslash{}hslayout}-Befehl platziert wird, so müssen
\emph{alle} Elemente mit \texttt{\textbackslash{}hslayout}
positioniert werden!!

Es gibt 4 mögliche Stellen:

\begin{itemize}\ttfamily
	\item frontmatter
	\item mainmatter
	\item backmatter
	\item end
\end{itemize}

Die Stelle \texttt{frontmatter} lässt das Element als Teil von
\texttt{\textbackslash{}hsfrontmatter} erscheinen.

Die Stelle \texttt{mainmatter} lässt das Element als Teil von
\texttt{\textbackslash{}hsmainmatter} erscheinen.

Die Stelle \texttt{backmatter} lässt das Element als Teil von
\texttt{\textbackslash{}hsbackmatter} erscheinen.

Die Stelle \texttt{end} lässt das Element am Ende des Dokumentes
erscheinen.

Aktuell stehen 8 Elemente zur Auswahl:

\begin{itemize}\ttfamily
	\item title
	\item restrictionnote
	\item \verb*!table of contents!
	\item \verb*!tables of figures/tables!
	\item abbreviations/glossary
	\item bibliography
	\item appendix
	\item index
\end{itemize}

Die Leerzeichen sind Teil der Bezeichner und unbedingt bei der
Verwendung zu übernehmen!!

Die Namen sind sprechend gewählt, dennoch folgen der Klarheit halber hier die
Erläuterungen der einzelnen Elemente:

\begin{description}
	\item[title] Die Titelei bestehend aus Titelblatt, Copyrightseite \& Widmung
	\item[restrictionnote] Sperrvermerk
	\item[table of contents] Inhaltsverzeichnis
	\item[tables of figures/tables] Abbildungs- und Tabellenverzeichnisse
	\item[abbreviations] Abkürzungen, Symbolverzeichnis, Glossar
	\item[bibliography] Literaturverzeichnis
	\item[appendix] keine Ausgabe, aber Ausgabeanpassungen für Anhänge
	\item[index] Index
\end{description}

Zur Klärung der Syntax hier noch die Standardeinstellungen von \HSKLbook:
\begin{verbatim*}
	\hslayout{frontmatter}{title}
	\hslayout{frontmatter}{restrictionnote}

	\hslayout{mainmatter}{table of contents}
	\hslayout{mainmatter}{tables of figures/tables}
	\hslayout{mainmatter}{abbreviations}

	\hslayout{backmatter}{bibliography}
	\hslayout{backmatter}{appendix}

	\hslayout{end}{index}
\end{verbatim*}
