\subsection{Neue Befehle}
\subsubsection{Titelei}
Die gesamte übliche Titelei wird automatisiert erstellt. Die nachfolgenden
Befehle dienen zur Anpassung von Namen, Thema und Ähnlichem in der Titelei. Die
eingetragenen Werte sind exemplarisch:

\paragraph{hsdedication}
Widmung oder Zitat\\
\textbackslash{}hsdedication\{Das Glück ist mit den Dummen.\}

\par\begingroup\small\begin{tabularx}{\textwidth}{>{\ttfamily}lX}
\parbox{
\widthof{\textbackslash{}hsdedication\{Das Glück ist mit den Dummen.\}} }
{\textbackslash{}hslogofile\{pfad/zum/logo\}}
		& Pfad zum Hochschullogo festlegen
	\\[3ex]
	\textbackslash{}hstitle\{Titelchen\}
		& Titel der Arbeit
	\\[3ex]
	\textbackslash{}hsdate\{10.~Januar~2038\}
		& Datum der Arbeit. Falls nicht angegeben wird das aktuelle Datum verwendet
%\end{tabularx}\endgroup
\\[3ex]
%\par\begingroup\small\begin{tabularx}{\textwidth}{>{\ttfamily}lX}
	\textbackslash{}hsyear\{2038\}
		& Jahr in dem die Arbeit geschrieben wurde. Falls nicht
		angegeben wird das aktuelle Jahr verwendet
	\\[3ex]
	\textbackslash{}hsmajor\{Studiengängchen\}
		& Studiengang
	\\[3ex]
	\textbackslash{}hsmatriculation\{0815123\}
		& Matrikelnummer
	\\[3ex]
	\textbackslash{}hsauthor\{Nämchen\}
		& Voller Name des Autors
	\\[3ex]
	\parbox{0.5\textwidth}{\textbackslash{}hsaddress\{Beispielstraße 1,
	12345 Teststadt\}}
		& \parbox{0.3\textwidth}{einzeilige Adresse\\~}
	\\[3ex]
	\textbackslash{}hspermanentcontact\{mailchen@example.com\}
		& Permanenter Kontakt wie z.B. E-Mail
	\\[3ex]
	\textbackslash{}hsplaceofstudy\{Studienörtchen\}
		& Studienort, also Zweibrücken, Pirmasens oder Kaiserslautern
	\\[3ex]
	\parbox{0.5\textwidth}{\textbackslash{}hsplaceofstudyaddress%
	\{Beispielstraße 1, 12345 Teststädtchen\}}
		& \parbox{0.35\textwidth}{Adresse des Studienorts, einzeilig\\~}
	%\\[3ex]
\end{tabularx}\endgroup\par\begingroup\small\begin{tabularx}{\textwidth}{>{\ttfamily}lX}
	\textbackslash{}hssupervisorname\{Nämelein\}
		& Name des Betreuers
	\\[3ex]
	\textbackslash{}hssupervisorphone\{+49 012 3456 7890\}
		& Telefonnummer des Betreuers
	\\[3ex]

	\textbackslash{}hssupervisormail\{naemchen@example.com\}
		& E-Mail des Betreuers
	\\[3ex]
	\textbackslash{}hsdistribution\{\textbackslash{}TeX\{\}lein\}
		& Verwendete LaTeX-Distribution. Meistens ist das
		\texttt{\textbackslash{}Mik\{\}TeX} oder
		\texttt{\textbackslash{}TeX\textasciitilde{}Live}
	\\[3ex]
	\textbackslash{}hsdedication\{Das Glück ist mit den Dummen.\}
		& Widmung oder Zitat
	\\[3ex]
	\textbackslash{}hskindofpaper\{Arbeitchen\}
		& Art der Arbeit: Bachelorarbeit, Praxisbericht, Hausarbeit etc.
	\\[3ex]
	\textbackslash{}hsworkenvironment\{Umgebüngchen\}
		& Ihre Entwicklungsumgebung bzw. Ihr Editor, zB
		\texttt{\textbackslash{}TeX\{\}nicCenter}, \texttt{Atom} oder
		\texttt{\textbackslash{}TeX\{\}works}
	\\[3ex]
	\textbackslash{}hsaddsupervisor\{Name2\}\{Telefon2\}\{Mail2\}
		& Einen weiteren Betreuer mit vollem Namen, Telefonnnummer und E-Mail
		hinzufügen
\end{tabularx}\endgroup

Wenn durch einzelne Optionen bestimme Ausgaben unterdrückt werden, so können die
Daten dazu (etwa copyright-showpermanentcontact und \verb!\hspermanentcontact!)
dennoch angegeben werden. So lässt sich bei Bedarf eine Option de-/aktivieren,
ohne dass weitere Änderungen am Dokument nötig sind.

\subsubsection{Sperrvermerk}
Nachfolgende Befehle dienen der Erstellung eines Sperrvermerks:

\par\begingroup\small\begin{tabularx}{\textwidth}{>{\ttfamily}lX}
	\textbackslash{}hscompany\{Firma GerneGroß\}
		& Name des Unternehmens mit Rechtsform
	\\[3ex]
	\textbackslash{}hsrestrictedchapters\{1, 2, 3, 5, 8, 12, Anhang B\}
		& betroffene Kapitel hier benennen (zB 1,4,7,12), auch die Anhänge
	\\[3ex]
	\textbackslash{}hsfaculty\{Fachbereichelchen\}
		& Fachbereich
	\\[3ex]
	\textbackslash{}hsrestrictionduration\{3\}
		& Dauer der Sperre in Jahren
\end{tabularx}\endgroup
